\documentclass{article}

% import packages
\usepackage{todonotes}
\usepackage[margin=1in]{geometry}
\setlength{\parskip}{2ex}
\setlength{\parindent}{0pt}
\usepackage{graphicx}
\usepackage{float} % use \begin{figure}[H] to force figure to be HERE
\graphicspath{{../figs/}}
\usepackage[margin=1cm,font=small,labelfont=bf]{caption}
\usepackage{subcaption}
\usepackage{scrextend} % for text definition labels
\addtokomafont{labelinglabel}{\sffamily}

% showing code
\usepackage{listings} % for showing programming code
\usepackage{color}
% color for code
\definecolor{codegreen}{rgb}{0,0.6,0}
\definecolor{codegray}{rgb}{0.5,0.5,0.5}
\definecolor{codepurple}{rgb}{0.58,0,0.82}
\definecolor{backcolour}{rgb}{0.95,0.95,0.92}
%style for code
\lstdefinestyle{pythonstyle}{
    backgroundcolor=\color{backcolour},   
    commentstyle=\color{codegreen},
    keywordstyle=\color{magenta},
    numberstyle=\tiny\color{codegray},
    stringstyle=\color{codepurple},
    basicstyle=\footnotesize,
    breakatwhitespace=false,         
    breaklines=true,                 
    captionpos=b,                    
    keepspaces=true,                 
    numbers=left,                    
    numbersep=5pt,                  
    showspaces=false,                
    showstringspaces=false,
    showtabs=false,                  
    tabsize=2
}
\lstset{style=pythonstyle}

% document info
\title{Introduction to AFM Line Scan Simulator}
\date{\today}
\author{Samuel Ng}

\begin{document}

% Title
\makeatletter
\begin{center}
  \Large
  \textbf{\@title}
  \par
  \@author
\end{center}
\makeatother


%------------------------------------------------------------
% Introduction
\section{Marcus Theory} \label{sec:introduction}


% Easy summary, lots of kinematics analogies so it's better suited for me:
% http://www.public.asu.edu/~laserweb/woodbury/classes/chm341/lecture_set8/The%20Marcus%20Theory%20of%20Electron%20Transfer.pdf

% Another summary with lots of figs:
% https://www.princeton.edu/chemistry/macmillan/group-meetings/VWS-Marcus.pdf

Marcus theory models the rate of thermally assisted adiabatic electron transfers. The rate equation for an electron tunneling from an initial site to a final site is given by:

\begin{equation}
  \nu_{i f} = \frac{|t_{i f}|^{2}}{\hbar} \sqrt{\frac{\pi}{\lambda k_{B} T}} \exp \left(- \frac{(\Delta G_{i f} + \lambda)^{2}}{4 \lambda k_{B} T} \right)
  \label{eq:marcus-tunneling-rate}
\end{equation}

Where $t_{i f}$ is the tunneling term (overlap of wavefunctions); $\lambda$ is the self-trapping energy; $\Delta G_{i f}$ is the Coulombic energy difference between the configurations before and after electron exchange. 




%------------------------------------------------------------
% Implementation
\section{AFM Line Scan Simulation} \label{sec:simulation}


\subsection{Implementation}

A tool for simulating AFM line scans has been implemented. The simulator takes the position of dangling bonds, scan type (read or write) and other parameters to produce simulated line scan plots.

The simulation engine consists of two classes:

\begin{itemize}
  \item AFMLine: This is the class that users interact with. It contains the user configuration for the AFM tip and dangling bonds, and facilitates the line scan simulation. To approximate an AFM's behavior, a measurement is made at each DB location to see whether it contains an electron. Plots are also generated by this class.
  \item MarcusModel: This class is called by \textit{AFMLine} to generate the timing for electron tunneling events. Marcus theory is used to determine electron hopping rates between two sites, which is used to generate randomized timings for electron tunneling events from an exponential distribution. This class is also responsible for keeping track of electron count and position.
\end{itemize}

Currently, two items are still pending proper implementation: 1. the current model of the AFM tip is a result of trial-and-error parameterization, Lucian's model should be able to fill this gap; 2. the electron population count is determined by hand, an automated implementation is needed.


\subsection{Usage}

Example simulation setup options can be found at \textit{afm-sim/src/python/demo.py}, executing it yields a line scan plot. The contents of \textit{demo.py} have also been included below for your convenience:

\lstinputlisting[language=Python]{../src/python/demo.py}

If you wish to alter the behavior of the simulation beyond what's shown above, you may tweak the `magic numbers' in \textit{afm-sim/src/python/marcus.py} within the section \textit{\# somewhat magic numbers}.




\end{document}
